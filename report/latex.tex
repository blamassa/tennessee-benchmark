\documentclass[a4paper, 12pt]{article}
\usepackage[utf8]{inputenc}

\title{Industrial IoT: the Tennessee benchmark study case}
\author{Breno Esteves Lamassa}
\date{\today}

\begin{document}
\maketitle

\section{Introduction}

- Motivation

- Previous works, cite the Fireman project

- Small bibliografic revision

\subsection{Tennessee Benchmark}

- Explain the process and the dataset provided

- Explain the flaws and why this benchmark exists

\subsection{Analysis of the chemical process as a system}

Apply the framework studied \textbf{PO Peculiar Operation, C1 Conditions of Production, C2 Conditions of Reproduction, C3 External Conditions}

\subsection{Information: characterization of the variables of the datasets}

Remember to list the types of data and types of information. If possible, classify all features on the dataset in those categories.

\section{Identification of Rare Events via IoT}

Cite again FIREMAN, explore some other resources on how Industrial IoT could be beneficial.

\subsection{Tools used}
Cite articles, the work of Gustavo at UFMG and that guy that proposed first the usage of PCA in this dataset (I think it's \textit{Russel et al.})

Cite also Machine Learning, big data and also other concepts presented at the work packages. Check the notes from the meetings and previous works from the members.


\subsection{Flaws identification via PCA and Hotelling stats}

Reproduce belisario et al.

\subsection{Flaws identification via Probability Theory}

Find the distribution that each of the feature belongs to \textbf{in the normal data.} Than, apply statistical tests for each variable in the flaws to check if it accusses any rare event.

\subsection{Other methods}
Based on the paper \textit{Data Mining and IoT}, apply some of the algorithmns available.

\section{IoT System proposal}

Design the system that will apply the rare events identification algorithms. Highlight that the idea of IoT is not only receiving data (the sensors), building a reality (structure of awareness) and providing the ability to act/interect/intervene on the system.

\subsection{Network}

Remember that the dataset only represents the sensors. Add to the network a receiving hub, decision making and processing node. Maybe even an actor in the system.

Understand the nature of the process. Maybe center the processing node in other subnodes related to the stage/timeline (boiler sensors, etc etc).

There are manipulated variables!! The/A processing node should be responsible in creating them.

***Too many nodes? Too much data connected to one node? WHat are the consequences???

\subsection{Reflexive-active systems}

Discuss the question above. What is the problems in having too many supporting realities? Is there loss of info? \textit{Interesting opportunity to apply Sheldon Entropy Theory}.

\subsection{Game Theory and Decision Making}



\end{document}

